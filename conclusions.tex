\chapter{Conclusions}
\label{chap:conclusions}

\begin{epigraphs}
\qitem{
  ``What dreadful nonsense we \emph{are} talking!''
}{---\textcite[255]{carroll-2004}}
\qitem{
  ``You may call it `nonsense' if you like, but \emph{I've} heard
  nonsense, compared with which that would be as sensible as a
  dictionary!''
}{---\textcite[173]{carroll-2004}}
\end{epigraphs}

Conclusions.

%% 1. Hacia atrás: categorías incompleto sin functores, functores
%% incompleto sin transformaciones naturales.

%% 2. Fin de los tres conceptos básicos de teoría de categorías. Con esto
%% queda abierto el mundo categórico.

%% 3. Hacia adelante: todo requiere transformaciones. Monoidals:
%% functores con trasnformaciones. Mónadas: endofunctores con
%% transformaciones.

%% 4. En transformaciones naturales: abre a parametricidad, visión
%% distinta de polimorfismo.

``Category theory can also be abused, and in several different styles.
One style of abuse is specious generality, in which some theory or
example is generalised in a way that does not actually include any new
examples of genuine interest. A related style of abuse is categorical
overkill, in which the language of category theory is used to describe
phenomena that do not actually require such an elaborate treatment or
terminology'' \parencite[50]{goguen-1991}.

``Category theory provides a number of broadly useful, and yet
surprisingly specific, guidelines for organising, generalising, and
discovering analogies among and within various branches of mathematics
and its applications. I wish to suggest that the existence of such
guidelines can be seen to support an alternative, more pragmatic view:
Foundations should provide general concepts and tools that reveal the
structures and interrelations of various areas of mathematics and its
applications, and that help in doing and using mathematics''
\parencite[64]{goguen-1991}. ``In a field which is not yet very well
developed, such as computing science, where it often seems that
getting the definitions right is the hardest task, foundations in this
sense can be very useful, because they can suggest which research
directions may be fruitful, using relatively explicit measures of
elegance and coherence. The succesful use of category theory for such
purposes suggests that it provides at least the beginnings of such a
foundation'' \parencite[64]{goguen-1991}.

With category theory, ``concepts may be dealt with at a higher level''
\parencite[xi]{pierce-1991}. ``Category theory formalizes otherwise
vague concepts'' \parencite[414]{poigne-1992}.

``Category-theoretic formulations of concepts can be more difficult to
grasp than their counterparts in other formalisms''
\parencite[xi]{pierce-1991}. ``One does not so much learn category
theory as absorb it over a period of time''
\parencite[25]{bird-demoor-1997}. ``From [116]: 'do not succumb to a
feeling that you must understand \emph{all} of category theory before
you put it to use'`` \parencite[xi]{pierce-1991}. ``Category theory
has been called 'abstract nonsense' by both its friends and its
detractors. Perhaps what this phrase suggests to both is that category
theory has relatively more form than content, compared to other areas
of mathematics'' \parencite[50]{goguen-1991}. Even though the
categorical formulation of some concepts may be difficult to
understand.

\section{Future Work}

TODO: The reasoning of Remark I deleted, which says that it is more
theoretically correct to not include a Functor constraint in the type
class declaration of monads, also applies for an
\texthaskell{Applicative} type constraint in the \texthaskell{Monad}
type class. In this case, it makes real practical sense to include
it.

Applicative functors: \parencite{mcbride-paterson-2008}

TODO: Regarding natural transformations, the relation between natural
transformations and parametrically polymorphic functions is not as
straightforward as discussed here. Polymorphic functions actually
correspond to lax natural transformations, but this fact is beyond the
scope of this dissertation.

TODO: Regarding Algebras, we are only concerned with inductive types,
not coinductive.

\paragraph{Scope}

This study of category theory is restricted to the following concepts:
This list is based on Mac Lane, Pierce, and Poigné.

\begin{enumerate}
\item Introductory concepts and infrastructure of category theory
  \begin{enumerate}
  \item Categories
  \item Infrastructure of category theory
    \begin{itemize}
    \item Monomorphisms, epimorphisms, and isomorphisms
    \item Initial and terminal objects
    \item Universal constructions
    \end{itemize}
  \end{enumerate}
\item Basic concepts of category theory
  \begin{enumerate}
  \item Functors
  \item Natural transformations
  \item Adjoints
  \item Limits
  \end{enumerate}
\item More specific concepts of category theory
  \begin{enumerate}
  \item Algebras
  \item Monads
  \end{enumerate}
\end{enumerate}

\clearemptydoublepage
